\documentclass[11pt,]{article}

\usepackage[margin=1in]{geometry}
%\usepackage{amscd}
%\usepackage{amssymb}
%\usepackage{amsmath}
%\usepackage{amsthm}
%\usepackage{enumitem}
%\usepackage{tikz}
%\usetikzlibrary{external}
%\tikzexternalize
%% This pre-compiles tikz pictures so that they do not have to be compiled every time.
%% For this to work you need to activate the flag:
%%   -shell-escape
%%
%% You can also manually run this by calling
%%   pdflatex -shell-escape filename.tex
%%

%\usepackage[colorlinks=true, linkcolor=blue, citecolor=blue]{hyperref}


\usepackage[backend=biber,
%style=trad-abbrv,
            backref=true,
            backrefstyle=three,
 %           hyperref=true,
	    url=false,
	    isbn=false,
	    doi=false,
	    sortcites]{biblatex}
\addbibresource{biblio_ConfESC.bib}

%\usepackage{pgfplots}
%\pgfplotsset{compat=1.16}
%\usepackage{wrapfig, caption}
%\usepackage{framed}
%\usepackage{float}
%\usetikzlibrary{arrows}
%\usepackage[font=small,labelfont=bf]{caption}
%\usepackage{xcolor}
%\usepackage{mathtools}
%\usepackage{ulem}
%\usepackage{comment}

%\usepackage[capitalise]{cleveref}
%\crefname{enumi}{}{}
%\usepackage{refcheck}
%%%making refcheck recognise cref
%%%% Infrastructure    
%\makeatletter
%\newcommand{\refcheckizelist}[1]{%
%  \expandafter\let\csname @@\string#1\endcsname#1%
%  \expandafter\DeclareRobustCommand\csname relax\string#1\endcsname[1]{%
%    \csname @@\string#1\endcsname{##1}% typeset as usual
%    \@for\rc@tmp:=##1\do{\wrtusdrf{\rc@tmp}}% loop over comma list
%  }%
%  \expandafter\let\expandafter#1\csname relax\string#1\endcsname
%}
%\makeatother
%
%
%
%
%%%% Now we add the reference commands we want refcheck to be aware of
%\refcheckizelist{\cref}
%\refcheckizelist{\Cref}
%
%

%\setlength{\parindent}{0pt} %--Uncomment to not indent paragraphs
%\linespread{2} %-- Uncomment for double spacing

%%%%%%%%%%%%%%%%%%%%%%%%%%%%%%%%%%%%%%%%%%%%%%%%%%%%%%%
%                   END OF MACROS                     %
%%%%%%%%%%%%%%%%%%%%%%%%%%%%%%%%%%%%%%%%%%%%%%%%%%%%%%%

\title{On exponential separation of analytic self-conformal sets on the real line}

\author{Bal\'azs B\'ar\'any\\(BME) \and Istv\'an
  Kolossv\'ary\\ (R\'enyi Institute) \and Sascha
Troscheit \\(Uppsala)}
\date{\today}


\begin{document}

\frenchspacing
\maketitle

%{\color{red}
%How can keywords, MSC appear at bottom footnote in this style? Should funding be taken to the end with an Acknowledgemnet section? \\
%2020 Mathematics Subject Classification.  Primary 28A80; Secondary 37C45 \\
%Key words and phrases. Analytic iterated function system, exponential separation condition,
%dimension drop conjecture, dual IFS, conjugation to self-similar IFS, ?? 
%}

\begin{abstract}
In a recent article, Rapaport showed that the there is no dimension drop for
exponentially separated analytic IFSs on the real line. We show that the set of such exponentially
separated IFSs in the space of analytic IFSs contains an open and dense set in the $\mathcal{C}^2$
topology. Moreover, we give a sufficient condition for the IFS to be exponentially separated which
allows us to construct explicit examples which are exponentially separated. The key
technical tool is the introduction of the \emph{dual IFS} which we believe has significant interest
in its own right. As an application we also characterise when an analytic IFS can be conjugated to a
self-similar IFS. 
\end{abstract}


%%%%%%%%%%%%%%%%%%%%%%%%%%%%%%%%%%%
%
%	INTRODUCTION
%
%%%%%%%%%%%%%%%%%%%%%%%%%%%%%%%%%%%

%%%%%%%%%%%%%%%%%%%%%%%%%%%%%%%%%%%%%%%%%%%%%%%%%%%%%%%%%%%%%%%%%%%%%%%%%%%
\section{Introduction and main results}
The geometric properties of attractors of iterated function systems have been extensively studied in recent
decades.
An iterated function system (IFS) $\Phi$ is a finite collection of strictly contracting self-maps
$(f_i)_{i\in\mathcal{I}}$ on a complete separable metric space $X$. By a result of
Hutchinson~\cite{Hutchinson_Attractor_81}, there exists a unique non-empty compact set $\Lambda$
satisfying the
invarianc

\printbibliography



\end{document}
