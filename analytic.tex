\documentclass[12pt,]{article}

\usepackage[margin=1in]{geometry}
\usepackage{amscd}
\usepackage{amssymb}
\usepackage{amsmath}
\usepackage{amsthm}
\usepackage{enumerate}
\usepackage{tikz}
\usetikzlibrary{external}
\tikzexternalize
%% This pre-compiles tikz pictures so that they do not have to be compiled every time. 
%% For this to work you need to activate the flag:
%%   -shell-escape
%%
%% You can also manually run this by calling
%%   pdflatex -shell-escape filename.tex
%%
\usepackage{pgfplots}
\pgfplotsset{compat=1.17}
\usepackage{wrapfig, framed, caption}
\usepackage{float}
\usetikzlibrary{arrows}
\usepackage[font=small,labelfont=bf]{caption}
\usepackage{xcolor}
\usepackage{mathtools}
\usepackage[colorlinks=true, linkcolor=blue, citecolor=blue,
pagebackref=true]{hyperref}
\usepackage{ulem}
\usepackage{comment}

\usepackage[capitalise]{cleveref}

%\usepackage{refcheck}
%%making refcheck recognise cref
%%% Infrastructure
%\makeatletter
%\newcommand{\refcheckize}[1]{%
%  \expandafter\let\csname @@\string#1\endcsname#1%
%  \expandafter\DeclareRobustCommand\csname relax\string#1\endcsname[1]{%
%  \csname @@\string#1\endcsname{##1}\@for\@temp:=##1\do{\wrtusdrf{\@temp}\wrtusdrf{{\@temp}}}}%
%  \expandafter\let\expandafter#1\csname relax\string#1\endcsname
%}

%%%

%%% Now we add the reference commands we want refcheck to be aware of
%\refcheckize{\cref}
%\refcheckize{\Cref}


\normalem
\parskip=5pt

%%%%%%%%%%%%%%%%%%%%%%%%%%%%%%%%%%%%%%%%%%%%%%%%%%%%%%%%%%%%%%%%
%%%     THEOREMS, STATEMENTS, DEFINITIONS  AND SO ON      %%%%%%
%%%%%%%%%%%%%%%%%%%%%%%%%%%%%%%%%%%%%%%%%%%%%%%%%%%%%%%%%%%%%%%%
\newtheorem{theorem}{Theorem}[section]
\newtheorem*{theorem*}{Theorem}
\newtheorem{theoremY}{Theorem Y}
\newtheorem*{theoremY*}{Theorem Y}
\newtheorem{theoremAB}{Theorem AB}
\newtheorem*{theoremAB*}{Theorem AB}

\newtheorem{corollary}[theorem]{Corollary}
\newtheorem*{corollary*}{Corollary}
\newtheorem{proposition}[theorem]{Proposition}
\newtheorem{lemma}[theorem]{Lemma}
\newtheorem{claim}[theorem]{Claim}
\newtheorem*{claim*}{Claim}
\newtheorem{conjecture}[theorem]{Conjecture}
\newtheorem{problem}[theorem]{Problem}
\newtheorem{question}[theorem]{Question}
\theoremstyle{definition}
\newtheorem{definition}[theorem]{Definition}
\theoremstyle{remark}
\newtheorem{remark}[theorem]{Remark}
\newtheorem*{remark*}{Remark}
\newtheorem{example}{Example}




%%%%%%%%%%%%%%%%%%%%%%%%%%%%%%%%%%%%%%%%%%%%%%%%%%%%%%%%%%%%%%%%
% THE DEFINITION OF A NEW FAMILY OF FONTS AND RELATED COMMANDS %
%%%%%%%%%%%%%%%%%%%%%%%%%%%%%%%%%%%%%%%%%%%%%%%%%%%%%%%%%%%%%%%%
%\font\tenmsy=msbm10 scaled 1200 \font\sevenmsy=msbm7 scaled 1200
%\font\fivemsy=msbm5 scaled 1200
%\newfam\msyfam
%\textfont\msyfam=\tenmsy \scriptfont\msyfam=\sevenmsy
%\scriptscriptfont\msyfam=\fivemsy
%\newcommand{\Bbb}[1]{{\fam\msyfam\relax#1}}
\renewcommand{\Bbb}[1]{\mathbb{#1}}
\newcommand{\bbA}{{\Bbb A}}         % often algebraic numbers
\newcommand{\bbB}{{\Bbb B}}
\newcommand{\bbC}{{\Bbb C}}         % complex numbers
\newcommand{\bbD}{{\Bbb D}}
\newcommand{\bbE}{{\Bbb E}}
\newcommand{\bbF}{{\Bbb F}}
\newcommand{\bbG}{{\Bbb G}}
\newcommand{\bbH}{{\Bbb H}}
\newcommand{\bbK}{{\Bbb K}}         % integer numbers
\newcommand{\bbL}{{\Bbb L}}
\newcommand{\bbM}{{\Bbb M}}
\newcommand{\bbN}{{\Bbb N}}         % natural numbers
\newcommand{\bbO}{{\Bbb O}}
\newcommand{\bbP}{{\Bbb P}}
\newcommand{\bbQ}{{\Bbb Q}}         % rational numbers
\newcommand{\bbR}{{\Bbb R}}        % real numbers
\newcommand{\bbRp}{{\Bbb R}^{+}}    % positive real numbers
\newcommand{\bbS}{{\Bbb S}}
\newcommand{\bbT}{{\Bbb T}}
\newcommand{\bbU}{{\Bbb U}}
\newcommand{\bbV}{{\Bbb V}}
\newcommand{\bbW}{\mathcal{W}}
\newcommand{\bbX}{{\Bbb X}}
\newcommand{\bbY}{{\Bbb Y}}
\newcommand{\bbZ}{{\Bbb Z}}         % integer numbers

%%%%%%%%%%%%%%%%%%%%%%%%%%%%%%%%%%%%%%%%%%%%%%%%%%%%%%%%%%%%%%%%
%%%%%%%%%%%%              \cal                      %%%%%%%%%%%%
%%%%%%%%%%%%%%%%%%%%%%%%%%%%%%%%%%%%%%%%%%%%%%%%%%%%%%%%%%%%%%%%
\newcommand{\cA}{{\cal A}}
\newcommand{\cB}{{\cal B}}
\newcommand{\cC}{{\cal C}}
\newcommand{\cD}{{\cal D}}
\newcommand{\cE}{{\cal E}}
\newcommand{\cF}{{\cal F}}
\newcommand{\cG}{{\cal G}}
\newcommand{\cH}{{\cal H}}
\newcommand{\cI}{{\cal I}}
\newcommand{\cJ}{{\cal J}}
\newcommand{\cK}{{\cal K}}
\newcommand{\cL}{{\cal L}}
\newcommand{\cM}{{\cal M}}
\newcommand{\cN}{{\cal N}}
\newcommand{\cO}{{\cal O}}
\newcommand{\cP}{{\cal P}}
\newcommand{\cQ}{{\cal Q}}
\newcommand{\cR}{{\cal R}}
\newcommand{\cS}{{\cal S}}
\newcommand{\cSM}{{\cal S}^*}
\newcommand{\cT}{{\cal T}}
\newcommand{\cU}{{\cal U}}
\newcommand{\cV}{{\cal V}}
\newcommand{\cW}{{\cal W}}
\newcommand{\cX}{{\cal X}}
\newcommand{\cY}{{\cal Y}}
\newcommand{\cZ}{{\cal Z}}

%%%%%%%%%%%%%%%%%%%%%%%%%%%%%%%%%%%%%%%%%%%%%%%%%%%%%%%
%                       GREEK                         %
%%%%%%%%%%%%%%%%%%%%%%%%%%%%%%%%%%%%%%%%%%%%%%%%%%%%%%%
\newcommand{\ve}{\varepsilon}
\newcommand{\Om}{\Omega}
\newcommand{\U}{\Upsilon}
\newcommand{\La}{\Lambda}

\newcommand{\tpsi}{\tilde\psi}
\newcommand{\tphi}{\tilde\phi}
\newcommand{\tU}{\tilde\Upsilon}

\newcommand{\Pn}{\Phi_n}
\newcommand{\Pm}{\Phi_m}
\newcommand{\Pj}{\Phi_j}
\newcommand{\pn}{\varphi_n}


\newcommand{\balpha}{\boldsymbol{\alpha}}
\newcommand{\bgamma}{\boldsymbol{\gamma}}

\newcommand{\fixx}{\Phi}
\newcommand{\fixy}{\Theta}

%%%%%%%%%%%%%%%%%%%%%%%%%%%%%%%%%%%%%%%%%%%%%%%%%%%%%%%
%                       VECTORS                       %
%%%%%%%%%%%%%%%%%%%%%%%%%%%%%%%%%%%%%%%%%%%%%%%%%%%%%%%

\newcommand{\x}{\mathbf{x}}
\newcommand{\y}{\mathbf{y}}
\newcommand{\p}{\mathbf{p}}
\newcommand{\q}{\mathbf{q}}
\newcommand{\br}{\mathbf{r}}
\newcommand{\bv}{\mathbf{v}}
\newcommand{\0}{\mathbf{0}}
\newcommand{\ba}{{\overline a}}
%%%%%%%%%%%%%%%%%%%%%%%%%%%%%%%%%%%%%%%%%%%%%%%%%%%%%%%
%                 VARIOUS COMMANDS                    %
%%%%%%%%%%%%%%%%%%%%%%%%%%%%%%%%%%%%%%%%%%%%%%%%%%%%%%%
\newcommand{\ie}{{\it i.e.}\/ }
\newcommand{\eg}{{\it e.g.}\/ }
\newcommand{\diam}{\text{diam}}
\newcommand{\dist}{\operatorname{dist}}
%\newcommand{\set}[1]{\left\{#1\right\}}
\newcommand{\Veronese}{\cV}
\renewcommand{\le}{\leq}
\renewcommand{\ge}{\geq}
\newcommand{\ra}{R_{\alpha}}
\newcommand{\kgb}{K_{G,B}}
\newcommand{\kgbl}{K_{G',B^{(l)}}}
\newcommand{\kgbf}{K^f_{G,B}}
\newcommand{\hf}{\cH^f}
\newcommand{\hs}{\cH^s}
\newcommand{\eps}{\varepsilon}

\newcommand{\cev}[1]{\reflectbox{\ensuremath{\vec{\reflectbox{\ensuremath{#1}}}}}}
\newcommand{\bi}{\vec{\imath}\,}
\newcommand{\bj}{\vec{\jmath}\,}
\newcommand{\bk}{{\vec{k}}}
\newcommand{\bo}{{\vec{o}}}
\newcommand{\bbi}{\,\cev{\imath}}
\newcommand{\bbj}{\,\cev{\jmath}}
\newcommand{\bbk}{{\cev{k}}}
\newcommand{\bbo}{{\cev{o}}}

\newcommand{\an}{(A;n)}
\newcommand{\ann}{(A';n')}
\newcommand{\Ln}{(L;n)}
\newcommand{\Lj}{(L;j)}
\newcommand{\aj}{(A;j)}
\newcommand{\ajj}{(A';j')}
\newcommand{\ajs}{(A;j^*)}
\newcommand{\aji}{(A;j_i)}
\newcommand{\ajis}{(A;j_{i^{**}})}
\newcommand{\as}{(A;s)}
\newcommand{\at}{(A;t)}
\newcommand{\can}{\cC\an}
\newcommand{\caj}{\cC\aj}
\newcommand{\cajs}{\cC\ajs}
\newcommand{\caji}{\cC\aji}
\newcommand{\cajis}{\cC\ajis}
\newcommand{\cas}{\cC\as}
\newcommand{\cat}{\cC\at}

\newcommand{\tW}{\widetilde{W}}

\newcommand{\Id}{\text{Id}}
\newcommand{\Exact}{\mathbf{Exact}}
\newcommand{\Bad}{\mathbf{Bad}}
\newcommand{\sing}{\mathbf{Sing}}
\newcommand{\DI}{\mathbf{DI}}
\newcommand{\FS}{\mathbf{FS}}
\newcommand{\ibad}{\mathbf{IBA}}
\newcommand{\well}{\mathbf{WA}}
\newcommand{\vwa}{\mathbf{VWA}}
\newcommand{\nvwa}{\mathbf{NVWA}}
\newcommand{\bone}{\boldsymbol{1}}
\newcommand{\recipso}{\mathcal R_0}
\newcommand{\freeD}{\mathcal{D}'}
\newcommand{\comp}{^{\mathsf{c}}}
\newcommand{\nopar}{{\parfillskip=0pt \par}}

\newcommand{\rfootnote}[1]{\footnote{\color{red}#1}}


\DeclarePairedDelimiter{\norm}{\lVert}{\rVert}
\DeclarePairedDelimiter{\abs}{\lvert}{\rvert}
\DeclarePairedDelimiter{\set}{\lbrace}{\rbrace}
\DeclarePairedDelimiter{\floor}{\lfloor}{\rfloor}
\DeclarePairedDelimiter{\ceil}{\lceil}{\rceil}
\DeclarePairedDelimiter{\parens}{\lparen}{\rparen}
\DeclarePairedDelimiter{\brackets}{\lbrack}{\rbrack}

\DeclareMathOperator{\Leb}{Leb}
\DeclareMathOperator{\sgn}{sgn}
\DeclareMathOperator{\dimh}{\dim_H}
\DeclareMathOperator{\vol}{vol}
\DeclareMathOperator{\Freq}{Freq}


\allowdisplaybreaks

%\setlength{\parindent}{0pt} %--Uncomment to not indent paragraphs
%\linespread{2} %-- Uncomment for double spacing

%%%%%%%%%%%%%%%%%%%%%%%%%%%%%%%%%%%%%%%%%%%%%%%%%%%%%%%
%                   END OF MACROS                     %
%%%%%%%%%%%%%%%%%%%%%%%%%%%%%%%%%%%%%%%%%%%%%%%%%%%%%%%

\title{Exponential separation for analytic self-conformal sets}

\author{Balazs Barany\footnote{Balazs has grants}\\(BME) \and Istvan Kolossvary\footnote{Istvan also
  has grants}\\ (Renyi Institute) \and Sascha
Troscheit\footnote{ST acknowledges travel funding from the Lisa \& Carl-Gustav Esseens Mathematics
Fund.} \\(Uppsala)}
\date{\today}
%\author{Author 1 \footnote{funding for Author 1} \\ Author 1 Affiliation \and Author 2 \\ Author 2 Affiliation}
%
%\date{\footnotesize{\it Dedication}}
%
\begin{document}

\frenchspacing
\maketitle

\begin{abstract}
  We show some stuff related to the dimension drop conjecture for analytical IFSs in the line.
\end{abstract}



%%%%%%%%%%%%%%%%%%%%%%%%%%%%%%%%%%%
%
%	INTRODUCTION
%
%%%%%%%%%%%%%%%%%%%%%%%%%%%%%%%%%%%


\section{Introduction} \label{sec:intro}
We first set up some notation. We consider an IFS $\Phi=\{f_i\}_{i=1}^N$ of maps $f_i:\bbR\to\bbR$
that are analytic $f_i \in C^\omega([-\eps,1+\eps])$ on $-\eps,1+\eps]$ for some, fixed, $\eps>0$.
For convenience we define $I=[0,1]$ and $I_{\eps}=[-\eps,1+\eps]$.
We assume that the maps $f_i$ are strictly contracting on $[0,1]$ and further that 
\[
  c_{\min} \leq \min_{x\in[0,1]}
  |f'(x)| \leq \max_{x\in[0,1]}|f'(x)| \leq c_{\max} 
\]
for some $0<c_{\min} \leq c_{\max}<1$.

We write $\Sigma = \{1,\dots, N\}^{\bbN}$ for all infinite words on the alphabet $\{1,\dots,N\}$ and
write $\Sigma_* = \bigcup_{k=0}^\infty \Sigma_k$ for all finite words, where $\Sigma_k =
\{1,\dots,N\}^k$ are words of length $k$.
We write $\bi=(i_1,i_2,i_3,\dots) \in\Sigma$ for specific infinite words and set $\bi|_n =
(i_1,i_2,\dots,i_n)\in \Sigma_n$ for words of length $n$. We write $[\bi|_n] = \{\bj\in\Sigma :
(j_1,\dots,j_n) = (i_1,\dotsi_n)\}$ for the cylinders of finite words $\bi|_n$. 

Given a finite word $\bi=(i_1,\dots,i_n)\in\Sigma_*$ we write
\[
  f_{\bi} = f_{i_1}\circ \dots \circ f_{i_n}.
\]
Similarly, for any given $\bi\in\Sigma_*$, we write $\bbi = (i_n,i_{n-1},\dots, i_1)$ and then
\[
  f_{\bbi} = f_{i_n}\circ \dots \circ f_{i_1}.
\]
We write $|\bi|$ for the length of $\bi$ and for $\bi,\bj\in\Sigma\cup\Sigma_*$ write $\bi\wedge\bj$
for the longest $\bk\in\Sigma\cup\Sigma_*$ such that $\bk=(k_1,\dots, k_n) = (i_1,\dots,
i_n)=(j_1,\dots,j_n)$.

\begin{definition}
  Let $\bi\in \Sigma\cup\Sigma_*$. We define the \emph{lineariser} $H_{\bi}$ by
  \[
    H_{\bi}(x) = \sum_{k=1}^{|\bi|}
  \frac{f''_{i_n}}{f'_{i_n}}(f_{\bbi|_{n-1}}(x))\cdot f'_{\bbi|_{n-1}}(x)
  \]
\end{definition}

\begin{definition}
  We say that the IFS $\Phi$ satisfies the \emph{strong exponential separation condition (SESC)} if
  there exists $c>0$ such that 
  \[
    |f_{\bi}(0)-f_{\bj}(0)| \geq c^n
  \]
  for all $\bi,\bj\in\Sigma_n$ and $n\in\bbN$, where $\bi\neq\bj$.
\end{definition}

Our main result is
\begin{theorem}
  \label{thm:main}
  If, for all distinct $\bi,\bj \in\Sigma\cup\Sigma_*$ we have
  \[
    \sup_{x\in[0,1]} |H_{\bi}(x) - H_{\bj}(x)| > 0,
  \]
  then $\Phi$ satisfies the SESC.
\end{theorem}






\section{A proof}\label{sec:proof}
We first observe that without loss of generality, we may assume $x\in[0,1]$ and
$0<\gamma<|f'_i(x)|<\lambda<1$.\rfootnote{generality? this just seems a definition of upper and lower
bounds?}
We first show
\begin{proposition}
  $H_{\bi}$ is analytic on $I_{\eps}$
\end{proposition}
\begin{proof}
  For all $x\in [-1,1]$ there exists an open complex neighbourhood $U \supseteq I$ such that
  $f_k(x)$ is analytic on $U$ and $|f_k'(z)|<\lambda$ for all $z\in U$ and $k\in\Sigma_1$.
  Hence $f_k(U) \subseteq U$ for all such $k$ and 
  \[
    \frac{f_k''}{f_k'},\quad f_{\bi}, \quad\text{and} \quad f'_{\bi}
  \]
  are analytic on $U$ for all $\bi\in\Sigma_*$ and $k\in\Sigma_1$.
  Since there exists $C>0$ such that 
  \[
\left|\frac{f_k''}{f_k'}\right| \leq C
  \]
  for all $k\in\Sigma_1$ and $z\in U$ by ??, we conclude that $H_{\bi|_n}$ is analytic on $U$ for
  all $\bi\in\Sigma$ and $n\in\bbN$. Further $H_{\bi|_n}$ converges uniformly to $H_{\bi}$ on $U$
  and $H_{\bi}$ is analytic on $U$ by Morera's theorem [??].
\end{proof}


\begin{lemma}\label{thm:analyticity}
  Let $f$ and $g$ be real analytic maps on $J\subseteq I$
  and let $\eta>0$ with $\sqrt{\eta}<|J|/2$.
  Denote
  \[
    C = \max\left\{\sup_{x\in J} |f''(x)|,\, \sup_{x\in J}|g''(x)\right\}.
  \]
  If $\sup_{x\in J} |f(x)-g(x)| \leq \eta$, then $\sup_{x\in J} |f'(x)-g'(x)|\leq (2+C)\sqrt{\eta}$.
\end{lemma}
\begin{proof}
  We compute
  \[
    f(x) = f(y) + f'(y)(x-y)+ \frac{f''(\xi)}{2}(x-y)^2,
  \]
  where $\xi\in(x,y)$.
  Let $y$ be arbitrary and let $x$ be such that $|x-y|=\sqrt{\eta}$ and $y\in J$.
  Then,
  \begin{align*}
    \eta &> |f(y)-f(x)| =
    \left|f(x)-g(x)+(f'(x)-g'(x))(x-y)+(f''(\xi_1)-g''(\xi_2))\frac{(x-y)^2}{2}\right|\\
	 &\geq |f'(x)-g'(x)|\cdot|x-y|-|f(x)-f(y)|-(|f''(\xi_1)|+|f''(\xi_2)|)\frac{(x-y)^2}{2}
  \end{align*}
  Thus,
  \[
    |f'(x)-g'(x)| \leq \frac{\eta+\eta+C \eta}{\sqrt{eta}} = (2+C)\sqrt{\eta}
  \]
  as required.
\end{proof}
A simple calculation shows that for every finite word $\bi\in\Sigma_*$, $H_{\bi}$ reduces to
\[
  H_{\bi}(x) = \frac{f''_{\bbi}(x)}{f'_{\bbi}(x)}.
\]
To simplify notation we will often write $f^{(k)}$ to refer to the $k$-th derivative of $f$.
Now let $g_k:\bbR^k\to \bbR$ be such that $g_1(x)=x$ and 
\[
  g_{k+1}(y_{k+1},\dots,y_1)=\sum_{l=1}^k (g_k)'_{y_l}(y_k,\dots,y_1)\cdot y_{l+1} +
  g_k(y_k,\dots,y_1)\cdot y_1.
\]
This follows from simple induction as
\[
  \frac{f_{\bi}^{(k+1)}}{f_{\bi}'(x)} - \frac{f_{\bi}^{(k)}}{f_{\bi}'(x)^2}\cdot f_{\bi}''(x)
  =\sum_{l=0}^{k-2}(g_k)'_{y_l}(H_{\bbi}^{(k-2)},\dots,H_{\bbi}(x)) \cdot H_{\bbi}^{(l+1)}(x).
\]
We note here that $g_k$ is a $k$-variable polynomial.

Recall Fa\`a di Bruno's formula:
\[
  (f\circ g)^{(k)}(x) = \sum_{\pi\in \Pi_k} f^{(|\pi|)}(g(x)) \cdot \prod_{B\in\pi}
  g^{(|B|)}(x),
\]
where $\Pi_k$ is the set of all partitions of $\{1,\dots,k\}$, and $B\in \pi$ refers to the
elements, or blocks, of the partition $\pi$.

Denote by $\phi_{i_k}(x):= \log|f'_{i_k}(x)|$. Then,
\[
  H_{\bi}(x) = \sum_{n=1}^\infty (\phi_{i_n}\circ f_{\bbi|_{n-1}})'(x)
\]
and 
\begin{align}
  H_{\bi}^{(k)}(x) 
  &=\sum_{n=1}^\infty (\phi_{i_n}\circ f_{\bbi|_{n-1}})^{(k+1)}(x) \nonumber\\
  &=\sum_{\pi\in\Pi_{k+1}} \sum_{n=1}^\infty \phi_{i_n}^{(|\pi|)}(f_{\bbi|_{n-1}}(x))\cdot
  \prod_{B\in\pi} f_{\bbi|_{n-1}}^{(|B|)}(x)\nonumber\\
  &=\sum_{\pi\in\Pi_{k+1}} \sum_{n=1}^\infty \phi_{i_n}^{(|\pi|)}(f_{\bbi|_{n-1}}(x))\cdot
  \prod_{B\in\pi}
  g_{|B|-1}(H_{\bi|_{n-1}}^{(|B|-2)}(x),\dots,H_{\bi|_{n-1}}(x))\cdot(f'_{\bbi|_{n-1}}(x))^{|\pi|}.\label{eq:ugly}
\end{align}
\begin{lemma}\label{thm:kbound}
  There exists $C_k$ such that for all $x\in [0,1)$ and all $\bi\in\Sigma$,
  \[
    |H_{\bi}^{(k)}(x)|\leq C_k
  \]
\end{lemma}
\begin{proof}
  By induction, let $D_k:=\max_{k\in\Sigma_1}\max_{x\in[0,1)} |\phi_k^{(k)}(x)|$ and so
  $|H_{\bi}(x)| \leq D_1/(1-\lambda)=:C_0$.
  Suppose that the statement is true for $k$, let $E_k:= \sup_{y_i\in[-C_i,C_i]}g_k(y_k,\dots,y_1)$.
  By \ref{eq:ugly},
  \[
    |H_{\bi}^{(k)}(x)| \leq \sum_{\pi\in\Pi_{k+1}}\sum_{n=1}^\infty D_{|\pi|} \cdot \prod_{B\in\pi}
    E_{|B|-1} \cdot \lambda^{(n-1)|\pi|}
    = \sum_{\pi\in\Pi_{k+1}} \frac{D_{|\pi|}\prod_{B\in\pi}E_{|B|-1}}{1-\lambda^{|\pi|}}
  \]
  and the statement follows.
\end{proof}
\begin{corollary}
  \label{thm:difcor}
  For all $k\geq 1$, for all $x,y\in I$ and for all $\bi\in\Sigma$,
  \[
    |H_{\bi}^{(k)}(x) - H_{\bi}^{(k)}(y)| \leq C_{k+1}\cdot |x-y|
  \]
\end{corollary}
\begin{lemma}\label{thm:difbound}
  For all $k\geq 0$ there exists $C_k>0$ such that 
  \[
    |H_{\bi}^{(k)}(x) - H_{\bj}^{(k)}(x)| \leq C_k \lambda^{|\bi\wedge\bj|}
  \]
  for all $x\in I$ and $\bi,\bj\in\Sigma$.
\end{lemma}
\begin{proof}
  Let $m = |\bi\wedge\bj|$. Again, by \cref{eq:ugly} and \cref{thm:kbound},
  \begin{align}
    &|H_{\bi}^{(k)}(x) - H_{\bj}^{(k)}(x)| 
    \nonumber\\
    =&
    \left|
    \sum_{\pi\in\Pi_{k+1}}\sum_{n=m+1}^\infty \phi_{i_n}^{(|\pi|)}(f_{\bbi|_{n-1}})\cdot
    \prod_{B\in\pi} g_{|B|-1}(H_{\bi|_{n-1}},\dots,H_{\bi|_{n-1}}(x)) \cdot
    f_{\bbi|_{n-1}}'(x)^{|\pi|}
  \right.
    \nonumber\\
    &
    \left. -
    \sum_{\pi\in\Pi_{k+1}}\sum_{n=m+1}^\infty \phi_{j_n}^{(|\pi|)}(f_{\bbj|_{n-1}})\cdot
    \prod_{B\in\pi} g_{|B|-1}(H_{\bj|_{n-1}},\dots,H_{\bj|_{n-1}}(x)) \cdot
    f_{\bbj|_{n-1}}'(x)^{|\pi|}
      \right|\nonumber\\
    &\leq
    \sum_{\pi\in\Pi_{k+1}}\sum_{n=m+1}^\infty 2D_{|\pi|}\cdot \prod_{B\in\pi} E_{|B|-1}
    \cdot\lambda^{n|\pi|} \leq C_{k+1} \lambda^m.\nonumber
  \end{align}
\end{proof}
We proceed with proving our main theorem, \cref{thm:main}.
\begin{proof}[Proof of \cref{thm:main}]
  Suppose that $\Phi$ is superexponentially concentrated, that is there exists a sequence
  $(\eta_n)_n$ such that $\log(\eta_n)/n\to-\infty$ and for all $n\in\bbN$ there exist distinct
  words $\bi,\bj\in\Sigma_n$ such that
  \[
    \sup_{x\in[0,1]}|f_{\bi}(x)-f_{\bj}(x)| \leq \eta_n.
  \]
  Since 
  \[
    |f_{\bi}''(x)| = |H_{\bi}(x)\cdot f_{\bi}'(x)|\leq C_0 \lambda^n
  \]
  we have
  \[
    \sup_{x\in[0,1]}|f_{\bi}'(x) - f_{\bj}'(x)| \leq (2+C_0 \lambda^n)\sqrt{\eta_n}
  \]
  and since
  \[
    |f_{\bi}'''(x)| = |g_2(H_{\bi}'(x),H_{\bi}(x))\cdot f_{\bi}'(x)| \leq E_2\cdot \lambda^n
  \]
  we have 
  \[
    \sup_{x\in[0,1]}|f_{\bi}''(x) - f_{\bj}''(x)| \leq (2+E_2\lambda^n)\sqrt{2+C_0 \lambda^n}\cdot
    \eta_n^{1/4}.
  \]
  We obtain
  \begin{align*}
    \left|\frac{f_{\bi}''(x)}{f_{\bi}'(x)} - \frac{f_{\bj}''(x)}{f_{\bj}'(x)}\right|
    &\leq
    \frac{|f_{\bi}''(x)|}{|f_{\bi}(x)||f_{\bj}'(x)|}\cdot|f_{\bi}'(x) - f_{\bj}'(x)|
    +\frac{1}{|f'_{\bj}(x)|} \cdot |f_{\bi}''(x) - f_{\bj}''(x)|
    \\
    & 
    \leq \frac{C_0}{\gamma^n}(2+C_0\lambda^n)\cdot \eta_n^{1/2} +\frac{1}{\gamma^n}(2+E_2
    \lambda^n)\sqrt{2+C_0 \lambda^n} \cdot \eta_n^{1/4} \;=:\;\eta_n'.
  \end{align*}
  For such $\bi\neq\bj$, let $m = \max\{m\leq n : i_m \neq j_m\}$. Then $\bi = \bi'\bk$ and $\bj =
  \bj'\bk$ with $|\bk| = m-1$ and $i_m\neq j_m$.

  By definition,
  \[
    H_{\bbi}(x) - H_{\bbj}(x) = f_{\bk}'(x)\cdot (H_{\bbi'}(f_{\bk}(x)) - H_{\bbj'}(f_{\bk}(x)))
  \]
  and so
  \[
    |H_{\bbi}(x) - H_{\bbj}(x)| \leq \frac{1}{|f'_n(x)|} \eta_n' \leq \gamma^{-n} \cdot \eta_n'
    =:\eta_n'' \leq \eta''_{|\bbi|}
  \]
  In particular, for every $n\geq 1$ there exists $1\leq m\leq n$, $\bi^{(n)},\bj^{(n)}\in\Sigma_m$,
  and $\bk^{(n)}\in\Sigma_{n-m}$ such that $i_1 \neq j_1$ and
  \[
    |H_{\bi^{(n)}}(f_{\bk^{(n)}}(x)) - H_{\bj^{(n)}}(f_{\bk^{(n)}}(x))| \leq \eta''_n
  \]
  for all $x\in I$.

  We now claim that $|\bk^{(n)}| \to \infty$ and argue by contradiction, i.e.\ we assume there
  exists a constant $C>0$ and infinitely many $n$ such that $\bk^{(n)} \leq C$. 
  Since $|\bi^{(n)}|+ |\bk^{(n)}| = |\bj^{(n)}+|\bk^{(n)}| = n$ we conclude, by compactness, that
  there exists a subsequence $n_l$ such that $\bi^{(n_l)} \to \bi^*\in\Sigma$,
  $\bj^{(n_l)}\to\bj^*\in\Sigma$ and $\bk^{(n_l)}\to \bk^*\in\Sigma$ with $i_1^*\neq j_1^*$.
  By \cref{thm:difbound} and \cref{thm:difcor} and that 
  \[
    H_{\bi^*}(f_{\bk^*}(x))\equiv H_{\bj^*}(f_{\bk^*}(x)) 
  \]
  for all $x\in I$, using analyticity gives $H_{\bi}(x) \equiv H_{\bj}(x)$ for all $x\in I$. This
  contradicts the main assumption.

  Now combining \cref{thm:analyticity} and \cref{thm:kbound} we conclude that for all $k$ there
  exists $\widetilde{C}_k>0$ such that for all $n\geq 1$ we have
  \[
    |H_{\bi^{(n)}}^{(k)}(f_{\bk^{(n)}}(x)) - H_{\bj^{(n)}}^{(k)}(f_{\bk^{(n)}}(x))|
  \leq \widetilde{C}_k \cdot \frac{1}{(f'_{\bk^{(n)}}(x))^k}\cdot\eta_n^{1/2^k}
  \]
  for all $x\in[0,1]$.
  By compactness, there exists $\bk^*\in\Sigma$ as well as $\bi^*,\bj^*\in\Sigma\cup\Sigma_*$ and a
  subsequence $n_l$ such that $\bi^{(n_l)}\to \bi^*$ and $\bj^{(n_l)}\to \bj^*$ as well as
  $f_{\bk^{(n_l)}}(x) \to \pi(\bk^*)$, where $i_1^*\neq j_1^*$ and
  $H_{\bi^*}^{(k)}(\pi(\bk^*))=H_{\bj^*}^{(k)}(\pi(\bk^*))$ for all $k$.
  Since $H_{\bi^*}$ and $H_{\bj^*}$ are analytic we get $H_{\bi^*}(x)\equiv H_{\bj^*}(x)$ for all
  $x\in[0,1]$. However, this claim also contradicts out main assumption, proving the main theorem.
\end{proof}

\begin{thebibliography}{99}
  \bibitem{Hutchinson}
  Hutchinson??

  \bibitem{Falconer}
  Falconer??

\end{thebibliography}

\end{document}
