\documentclass[12pt,]{article}

\usepackage[margin=1in]{geometry}
\usepackage{amscd}
\usepackage{amssymb}
\usepackage{amsmath}
\usepackage{amsthm}
\usepackage{enumerate}
%\usepackage{tikz}
%\usetikzlibrary{external}
%\tikzexternalize
%% This pre-compiles tikz pictures so that they do not have to be compiled every time. 
%% For this to work you need to activate the flag:
%%   -shell-escape
%%
%% You can also manually run this by calling
%%   pdflatex -shell-escape filename.tex
%%
\usepackage{pgfplots}
\pgfplotsset{compat=1.17}
\usepackage{wrapfig, framed, caption}
\usepackage{float}
\usetikzlibrary{arrows}
\usepackage[font=small,labelfont=bf]{caption}
\usepackage{xcolor}
\usepackage{mathtools}
\usepackage[colorlinks=true, linkcolor=blue, citecolor=blue,
pagebackref=true]{hyperref}
\usepackage{ulem}
\usepackage{comment}

\usepackage[capitalise]{cleveref}

\usepackage{refcheck}
%making refcheck recognise cref
%% Infrastructure
\makeatletter
\newcommand{\refcheckize}[1]{%
  \expandafter\let\csname @@\string#1\endcsname#1%
  \expandafter\DeclareRobustCommand\csname relax\string#1\endcsname[1]{%
  \csname @@\string#1\endcsname{##1}\@for\@temp:=##1\do{\wrtusdrf{\@temp}\wrtusdrf{{\@temp}}}}%
  \expandafter\let\expandafter#1\csname relax\string#1\endcsname
}

%%%

%%% Now we add the reference commands we want refcheck to be aware of
\refcheckize{\cref}
\refcheckize{\Cref}


\normalem
\parskip=5pt

%%%%%%%%%%%%%%%%%%%%%%%%%%%%%%%%%%%%%%%%%%%%%%%%%%%%%%%%%%%%%%%%
%%%     THEOREMS, STATEMENTS, DEFINITIONS  AND SO ON      %%%%%%
%%%%%%%%%%%%%%%%%%%%%%%%%%%%%%%%%%%%%%%%%%%%%%%%%%%%%%%%%%%%%%%%
\newtheorem{theorem}{Theorem}[section]
\newtheorem*{theorem*}{Theorem}
\newtheorem{theoremY}{Theorem Y}
\newtheorem*{theoremY*}{Theorem Y}
\newtheorem{theoremAB}{Theorem AB}
\newtheorem*{theoremAB*}{Theorem AB}

\newtheorem{corollary}[theorem]{Corollary}
\newtheorem*{corollary*}{Corollary}
\newtheorem{proposition}[theorem]{Proposition}
\newtheorem{lemma}[theorem]{Lemma}
\newtheorem{claim}[theorem]{Claim}
\newtheorem*{claim*}{Claim}
\newtheorem{conjecture}[theorem]{Conjecture}
\newtheorem{problem}[theorem]{Problem}
\newtheorem{question}[theorem]{Question}
\theoremstyle{definition}
\newtheorem{definition}[theorem]{Definition}
\theoremstyle{remark}
\newtheorem{remark}[theorem]{Remark}
\newtheorem*{remark*}{Remark}
\newtheorem{example}{Example}




%%%%%%%%%%%%%%%%%%%%%%%%%%%%%%%%%%%%%%%%%%%%%%%%%%%%%%%%%%%%%%%%
% THE DEFINITION OF A NEW FAMILY OF FONTS AND RELATED COMMANDS %
%%%%%%%%%%%%%%%%%%%%%%%%%%%%%%%%%%%%%%%%%%%%%%%%%%%%%%%%%%%%%%%%
%\font\tenmsy=msbm10 scaled 1200 \font\sevenmsy=msbm7 scaled 1200
%\font\fivemsy=msbm5 scaled 1200
%\newfam\msyfam
%\textfont\msyfam=\tenmsy \scriptfont\msyfam=\sevenmsy
%\scriptscriptfont\msyfam=\fivemsy
%\newcommand{\Bbb}[1]{{\fam\msyfam\relax#1}}
\renewcommand{\Bbb}[1]{\mathbb{#1}}
\newcommand{\bbA}{{\Bbb A}}         % often algebraic numbers
\newcommand{\bbB}{{\Bbb B}}
\newcommand{\bbC}{{\Bbb C}}         % complex numbers
\newcommand{\bbD}{{\Bbb D}}
\newcommand{\bbE}{{\Bbb E}}
\newcommand{\bbF}{{\Bbb F}}
\newcommand{\bbG}{{\Bbb G}}
\newcommand{\bbH}{{\Bbb H}}
\newcommand{\bbK}{{\Bbb K}}         % integer numbers
\newcommand{\bbL}{{\Bbb L}}
\newcommand{\bbM}{{\Bbb M}}
\newcommand{\bbN}{{\Bbb N}}         % natural numbers
\newcommand{\bbO}{{\Bbb O}}
\newcommand{\bbP}{{\Bbb P}}
\newcommand{\bbQ}{{\Bbb Q}}         % rational numbers
\newcommand{\bbR}{{\Bbb R}}        % real numbers
\newcommand{\bbRp}{{\Bbb R}^{+}}    % positive real numbers
\newcommand{\bbS}{{\Bbb S}}
\newcommand{\bbT}{{\Bbb T}}
\newcommand{\bbU}{{\Bbb U}}
\newcommand{\bbV}{{\Bbb V}}
\newcommand{\bbW}{\mathcal{W}}
\newcommand{\bbX}{{\Bbb X}}
\newcommand{\bbY}{{\Bbb Y}}
\newcommand{\bbZ}{{\Bbb Z}}         % integer numbers

%%%%%%%%%%%%%%%%%%%%%%%%%%%%%%%%%%%%%%%%%%%%%%%%%%%%%%%%%%%%%%%%
%%%%%%%%%%%%              \cal                      %%%%%%%%%%%%
%%%%%%%%%%%%%%%%%%%%%%%%%%%%%%%%%%%%%%%%%%%%%%%%%%%%%%%%%%%%%%%%
\newcommand{\cA}{{\cal A}}
\newcommand{\cB}{{\cal B}}
\newcommand{\cC}{{\cal C}}
\newcommand{\cD}{{\cal D}}
\newcommand{\cE}{{\cal E}}
\newcommand{\cF}{{\cal F}}
\newcommand{\cG}{{\cal G}}
\newcommand{\cH}{{\cal H}}
\newcommand{\cI}{{\cal I}}
\newcommand{\cJ}{{\cal J}}
\newcommand{\cK}{{\cal K}}
\newcommand{\cL}{{\cal L}}
\newcommand{\cM}{{\cal M}}
\newcommand{\cN}{{\cal N}}
\newcommand{\cO}{{\cal O}}
\newcommand{\cP}{{\cal P}}
\newcommand{\cQ}{{\cal Q}}
\newcommand{\cR}{{\cal R}}
\newcommand{\cS}{{\cal S}}
\newcommand{\cSM}{{\cal S}^*}
\newcommand{\cT}{{\cal T}}
\newcommand{\cU}{{\cal U}}
\newcommand{\cV}{{\cal V}}
\newcommand{\cW}{{\cal W}}
\newcommand{\cX}{{\cal X}}
\newcommand{\cY}{{\cal Y}}
\newcommand{\cZ}{{\cal Z}}

%%%%%%%%%%%%%%%%%%%%%%%%%%%%%%%%%%%%%%%%%%%%%%%%%%%%%%%
%                       GREEK                         %
%%%%%%%%%%%%%%%%%%%%%%%%%%%%%%%%%%%%%%%%%%%%%%%%%%%%%%%
\newcommand{\ve}{\varepsilon}
\newcommand{\Om}{\Omega}
\newcommand{\U}{\Upsilon}
\newcommand{\La}{\Lambda}

\newcommand{\tpsi}{\tilde\psi}
\newcommand{\tphi}{\tilde\phi}
\newcommand{\tU}{\tilde\Upsilon}

\newcommand{\Pn}{\Phi_n}
\newcommand{\Pm}{\Phi_m}
\newcommand{\Pj}{\Phi_j}
\newcommand{\pn}{\varphi_n}


\newcommand{\balpha}{\boldsymbol{\alpha}}
\newcommand{\bgamma}{\boldsymbol{\gamma}}

\newcommand{\fixx}{\Phi}
\newcommand{\fixy}{\Theta}

%%%%%%%%%%%%%%%%%%%%%%%%%%%%%%%%%%%%%%%%%%%%%%%%%%%%%%%
%                       VECTORS                       %
%%%%%%%%%%%%%%%%%%%%%%%%%%%%%%%%%%%%%%%%%%%%%%%%%%%%%%%

\newcommand{\x}{\mathbf{x}}
\newcommand{\y}{\mathbf{y}}
\newcommand{\p}{\mathbf{p}}
\newcommand{\q}{\mathbf{q}}
\newcommand{\br}{\mathbf{r}}
\newcommand{\bv}{\mathbf{v}}
\newcommand{\0}{\mathbf{0}}
\newcommand{\ba}{{\overline a}}
%%%%%%%%%%%%%%%%%%%%%%%%%%%%%%%%%%%%%%%%%%%%%%%%%%%%%%%
%                 VARIOUS COMMANDS                    %
%%%%%%%%%%%%%%%%%%%%%%%%%%%%%%%%%%%%%%%%%%%%%%%%%%%%%%%
\newcommand{\ie}{{\it i.e.}\/ }
\newcommand{\eg}{{\it e.g.}\/ }
\newcommand{\diam}{\text{diam}}
\newcommand{\dist}{\operatorname{dist}}
%\newcommand{\set}[1]{\left\{#1\right\}}
\newcommand{\Veronese}{\cV}
\renewcommand{\le}{\leq}
\renewcommand{\ge}{\geq}
\newcommand{\ra}{R_{\alpha}}
\newcommand{\kgb}{K_{G,B}}
\newcommand{\kgbl}{K_{G',B^{(l)}}}
\newcommand{\kgbf}{K^f_{G,B}}
\newcommand{\hf}{\cH^f}
\newcommand{\hs}{\cH^s}
\newcommand{\eps}{\varepsilon}

\newcommand{\cev}[1]{\reflectbox{\ensuremath{\vec{\reflectbox{\ensuremath{#1}}}}}}
\newcommand{\bi}{\vec{\imath}\,}
\newcommand{\bj}{\vec{\jmath}\,}
\newcommand{\bk}{{\vec{k}}}
\newcommand{\bu}{{\vec{u}}}
\newcommand{\bo}{{\vec{o}}}
\newcommand{\bbi}{\,\cev{\imath}}
\newcommand{\bbj}{\,\cev{\jmath}}
\newcommand{\bbu}{{\cev{u}}}
\newcommand{\bbo}{{\cev{o}}}

\newcommand{\an}{(A;n)}
\newcommand{\ann}{(A';n')}
\newcommand{\Ln}{(L;n)}
\newcommand{\Lj}{(L;j)}
\newcommand{\aj}{(A;j)}
\newcommand{\ajj}{(A';j')}
\newcommand{\ajs}{(A;j^*)}
\newcommand{\aji}{(A;j_i)}
\newcommand{\ajis}{(A;j_{i^{**}})}
\newcommand{\as}{(A;s)}
\newcommand{\at}{(A;t)}
\newcommand{\can}{\cC\an}
\newcommand{\caj}{\cC\aj}
\newcommand{\cajs}{\cC\ajs}
\newcommand{\caji}{\cC\aji}
\newcommand{\cajis}{\cC\ajis}
\newcommand{\cas}{\cC\as}
\newcommand{\cat}{\cC\at}

\newcommand{\tW}{\widetilde{W}}

\newcommand{\Id}{\text{Id}}
\newcommand{\Exact}{\mathbf{Exact}}
\newcommand{\Bad}{\mathbf{Bad}}
\newcommand{\sing}{\mathbf{Sing}}
\newcommand{\DI}{\mathbf{DI}}
\newcommand{\FS}{\mathbf{FS}}
\newcommand{\ibad}{\mathbf{IBA}}
\newcommand{\well}{\mathbf{WA}}
\newcommand{\vwa}{\mathbf{VWA}}
\newcommand{\nvwa}{\mathbf{NVWA}}
\newcommand{\bone}{\boldsymbol{1}}
\newcommand{\recipso}{\mathcal R_0}
\newcommand{\freeD}{\mathcal{D}'}
\newcommand{\comp}{^{\mathsf{c}}}
\newcommand{\nopar}{{\parfillskip=0pt \par}}

\newcommand{\rfootnote}[1]{\footnote{\color{red}#1}}


\DeclarePairedDelimiter{\norm}{\lVert}{\rVert}
\DeclarePairedDelimiter{\abs}{\lvert}{\rvert}
\DeclarePairedDelimiter{\set}{\lbrace}{\rbrace}
\DeclarePairedDelimiter{\floor}{\lfloor}{\rfloor}
\DeclarePairedDelimiter{\ceil}{\lceil}{\rceil}
\DeclarePairedDelimiter{\parens}{\lparen}{\rparen}
\DeclarePairedDelimiter{\brackets}{\lbrack}{\rbrack}

\DeclareMathOperator{\Leb}{Leb}
\DeclareMathOperator{\sgn}{sgn}
\DeclareMathOperator{\dimh}{\dim_H}
\DeclareMathOperator{\vol}{vol}
\DeclareMathOperator{\Freq}{Freq}


\allowdisplaybreaks

%\setlength{\parindent}{0pt} %--Uncomment to not indent paragraphs
%\linespread{2} %-- Uncomment for double spacing

%%%%%%%%%%%%%%%%%%%%%%%%%%%%%%%%%%%%%%%%%%%%%%%%%%%%%%%
%                   END OF MACROS                     %
%%%%%%%%%%%%%%%%%%%%%%%%%%%%%%%%%%%%%%%%%%%%%%%%%%%%%%%

\title{Exponential separation for analytic self-conformal sets}

\author{Balazs Barany\footnote{Balazs has grants}\\(BME) \and Istvan Kolossvary\footnote{Istvan also
  has grants}\\ (Renyi Institute) \and Sascha
Troscheit\footnote{ST acknowledges travel funding from the Lisa \& Carl-Gustav Esseens Mathematics
Fund.} \\(Uppsala)}
\date{\today}
%\author{Author 1 \footnote{funding for Author 1} \\ Author 1 Affiliation \and Author 2 \\ Author 2 Affiliation}
%
%\date{\footnotesize{\it Dedication}}
%
\begin{document}

\frenchspacing
\maketitle

\begin{abstract}
  We show some stuff related to the dimension drop conjecture for analytical IFSs in the line.
\end{abstract}



%%%%%%%%%%%%%%%%%%%%%%%%%%%%%%%%%%%
%
%	INTRODUCTION
%
%%%%%%%%%%%%%%%%%%%%%%%%%%%%%%%%%%%


\section{Introduction} \label{sec:intro}
We first set up some notation. We consider an IFS $\Phi=\{f_i\}_{i=1}^N$ of maps $f_i:\bbR\to\bbR$
that are analytic $f_i \in C^\omega([-\eps,1+\eps])$ on $I_{\eps}=[-\eps,1+\eps]$ for some, fixed, $\eps>0$.
For convenience we also write define $I=[0,1]$.
We assume that the maps $f_i$ are strictly contracting on $[0,1]$ and further that 
\[
  c_{\min} < \min_{x\in[0,1]}
  |f'(x)| \leq \max_{x\in[0,1]}|f'(x)| < c_{\max} 
\]
for some $0<c_{\min} \leq c_{\max}<1$.

We write $\Sigma = \{1,\dots, N\}^{\bbN}$ for all infinite words on the alphabet $\{1,\dots,N\}$ and
write $\Sigma_* = \bigcup_{k=0}^\infty \Sigma_k$ for all finite words, where $\Sigma_k =
\{1,\dots,N\}^k$ are words of length $k$.
We write $\bi=(i_1,i_2,i_3,\dots) \in\Sigma$ for specific infinite words and set $\bi|_n =
(i_1,i_2,\dots,i_n)\in \Sigma_n$ for words of length $n$. We write $[\bi|_n] = \{\bj\in\Sigma :
(j_1,\dots,j_n) = (i_1,\dots, i_n)\}$ for the cylinders of finite words $\bi|_n$. 

Given a finite word $\bi=(i_1,\dots,i_n)\in\Sigma_*$ we write
\[
  f_{\bi} = f_{i_1}\circ \dots \circ f_{i_n}.
\]
Similarly, for any given $\bi\in\Sigma_*$, we write $\bbi = (i_n,i_{n-1},\dots, i_1)$ and then
\[
  f_{\bbi} = f_{i_n}\circ \dots \circ f_{i_1}.
\]
We write $|\bi|$ for the length of $\bi$ and for $\bi,\bj\in\Sigma\cup\Sigma_*$ write $\bi\wedge\bj$
for the longest $\bu\in\Sigma\cup\Sigma_*$ such that $\bu=(k_1,\dots, k_n) = (i_1,\dots,
i_n)=(j_1,\dots,j_n)$.

\begin{definition}
  Let $\bi\in \Sigma\cup\Sigma_*$. We define the \emph{protolineariser(?)} $H_{\bi}$ by
  \[
    H_{\bi}(x) = \sum_{n=1}^{|\bi|}
  \frac{f''_{i_n}}{f'_{i_n}}(f_{\bbi|_{n-1}}(x))\cdot f'_{\bbi|_{n-1}}(x)
  \]
\end{definition}

\begin{definition}
  We say that the IFS $\Phi$ satisfies the \emph{strong exponential separation condition (SESC)} if
  there exists $c>0$ such that 
  \[
    |f_{\bi}(0)-f_{\bj}(0)| \geq c^n
  \]
  for all $\bi,\bj\in\Sigma_n$ and $n\in\bbN$, where $\bi\neq\bj$.
\end{definition}

We write $\phi_{i_k}(x):= \log|f'_{i_k}(x)|$ and get, equivalently,
\begin{equation}
  H_{\bi}(x) = \sum_{n=1}^\infty (\phi_{i_n}\circ f_{\bbi|_{n-1}})'(x)
  \label{eq:alternativeH}
\end{equation}

Our main result is
\begin{theorem}
  \label{thm:main}
  If, for all distinct $\bi,\bj \in\Sigma$ we have
  \[
    \sup_{x\in[0,1]} |H_{\bi}(x) - H_{\bj}(x)| > 0,
  \]
  then $\Phi$ satisfies the SESC.
\end{theorem}






\section{A proof}\label{sec:proof}
We first show
\begin{proposition}
  $H_{\bi}$ is analytic on $I_{\eps}$.
\end{proposition}
\begin{proof}
  There exists an open bounded complex neighbourhood $U \supseteq I$ such that
  $z\mapsto f_j(z)$ is analytic on $U$ and $c_{\min}<|f_j'(z)|<c_{\max}$ for all $z\in U$ and $j\in\Sigma_1$.
  Hence $f_j(U) \subseteq U$ and 
  \[
    \frac{f_j''}{f_j'},\quad f_{\bi}, \quad\text{and} \quad f'_{\bi}
  \]
  are analytic on $U$ for all $\bi\in\Sigma_*$ and $j\in\Sigma_1$.
  Since there exists $C>0$ such that 
  \[
\left|\frac{f_j''}{f_j'}\right| \leq C
  \]
  for all $j\in\Sigma_1$ and $z\in U$, we conclude that $H_{\bi|_n}$ is analytic on $U$ for
  all $\bi\in\Sigma$ and $n\in\bbN$. Further $H_{\bi|_n}$ converges uniformly to $H_{\bi}$ on $U$
and $H_{\bi}$ is analytic on $U$ by Morera's theorem \cite[Theorem 10.17]{Rudin87}.
\end{proof}


\begin{lemma}\label{thm:analyticity}
  Let $f$ and $g$ be real analytic maps on $J$
  and let $\eta>0$ with $2\sqrt{\eta}<|J|$.
  Denote
  \[
    Q = \max\left\{\sup_{x\in J} |f''(x)|,\, \sup_{x\in J}|g''(x)|\right\}.
  \]
  If $\sup_{x\in J} |f(x)-g(x)| \leq \eta$, then $\sup_{x\in J} |f'(x)-g'(x)|\leq (2+Q)\sqrt{\eta}$.
\end{lemma}
\begin{proof}
  We compute
  \[
    f(y) = f(x) + f'(x)(y-x)+ \frac{f''(\xi)}{2}(y-x)^2,
  \]
  where $\xi\in(x,y)$.
  Let $x$ be arbitrary and take $y\in J$ such that $|x-y|=\sqrt{\eta}$.
  Then,
  \begin{align*}
    \eta &> |f(y)-g(y)| =
    \left|f(x)-g(x)+(f'(x)-g'(x))(y-x)+(f''(\xi_1)-g''(\xi_2))\frac{(y-x)^2}{2}\right|\\
	 &\geq |f'(x)-g'(x)|\cdot|y-x|-|f(x)-g(x)|-(|f''(\xi_1)|+|g''(\xi_2)|)\frac{(y-x)^2}{2}
  \end{align*}
  Thus,
  \[
    |f'(x)-g'(x)| \leq \frac{\eta+\eta+Q \eta}{\sqrt{\eta}} = (2+Q)\sqrt{\eta}
  \]
  as required.
\end{proof}

To simplify notation we will often write $f^{(k)}$ to refer to the $k$-th derivative of $f$.
A simple calculation now shows that for every finite word $\bi\in\Sigma_*$, $H_{\bi}$ reduces to
\[
  H_{\bi}(x) = \frac{f''_{\bbi}(x)}{f'_{\bbi}(x)}.
\]
Now let $g_k:\bbR^k\to \bbR$ be such that $g_1(x)=x$ and 
\[
  g_{k+1}(y_{k+1},\dots,y_1)=\sum_{l=1}^k (g_k)'_{y_l}(y_k,\dots,y_1)\cdot y_{l+1} +
  g_k(y_k,\dots,y_1)\cdot y_1.
\]
We note here that $g_k$ is a $k$-variable polynomial.

Recall Fa\`a di Bruno's formula:
\[
  (f\circ g)^{(k)}(x) = \sum_{\pi\in \Pi_k} f^{(|\pi|)}(g(x)) \cdot \prod_{B\in\pi}
  g^{(|B|)}(x),
\]
where $\Pi_k$ is the set of all partitions of $\{1,\dots,k\}$, and $B\in \pi$ refers to the
elements, or blocks, of the partition $\pi$.


\begin{lemma}
  \label{thm:ugly}
  The $k$-th derivative of the preprotolineariserything(?) is
  \[
  H_{\bi}^{(k)}(x) 
  =\sum_{\pi\in\Pi_{k+1}} \sum_{n=1}^\infty \phi_{i_n}^{(|\pi|)}(f_{\bbi|_{n-1}}(x))\cdot
  \prod_{B\in\pi}
  g_{|B|-1}(H_{\bi|_{n-1}}^{(|B|-2)}(x),\dots,H_{\bi|_{n-1}}(x))\cdot(f'_{\bbi|_{n-1}}(x))^{|\pi|}.
  \]
\end{lemma}
\begin{proof}
  We first show
  \begin{equation}
    \label{eq:fandg}
    \frac{f_{\bi}^{(k)}(x)}{f_{\bi}'(x)}
    =
    g_{k-1}(H_{\bbi}^{(k-2)}(x), \dots, H_{\bbi}(x)).
  \end{equation}
  Taking derivatives,
  \[
    \frac{f_{\bi}^{(k+1)}(x)}{f_{\bi}'(x)} - \frac{f_{\bi}^{(k)}(x)}{(f_{\bi}'(x))^2}\cdot f_{\bi}''(x)
    =\sum_{l=0}^{k-2}(g_{k-1})'_{y_l}(H_{\bbi}^{(k-2)},\dots,H_{\bbi}(x)) \cdot H_{\bbi}^{(l+1)}(x).
  \]
  We will argue by induction. By the inductive hypothesis for $k$, the second summand of the left hand
  side reduces to
  \[
    \frac{f_{\bi}^{(k)}(x)}{f_{\bi}'(x)}\cdot \frac{f_{\bi}''(x)}{f_{\bi}(x)}
    =g_{k-1}(H_{\bbi}^{(k-2)}(x),\dots, H_{\bbi}(x)) \cdot H_{\bbi}(x).
  \]
  Rearranging proves the inductive step,
  \begin{align*}
    \frac{f_{\bi}^{(k+1)}(x)}{f_{\bi}'(x)}
  &=
  g_{k-1}(H_{\bbi}^{(k-2)}(x),\dots, H_{\bbi}(x)) \cdot H_{\bbi}(x)
  +
  \sum_{l=0}^{k-2}(g_{k-1})'_{y_l}(H_{\bbi}^{(k-2)},\dots,H_{\bbi}(x)) \cdot H_{\bbi}^{(l+1)}(x)
  \\
  &=
  g_{k}(H_{\bbi}^{(k-1)}(x), \dots, H_{\bbi}(x)).
  \end{align*}

Using \cref{eq:alternativeH,eq:fandg}, and Fa\`a di Bruno's formula,
\begin{align}
  H_{\bi}^{(k)}(x) 
  &=\sum_{n=1}^\infty (\phi_{i_n}\circ f_{\bbi|_{n-1}})^{(k+1)}(x) \nonumber\\
  &=\sum_{\pi\in\Pi_{k+1}} \sum_{n=1}^\infty \phi_{i_n}^{(|\pi|)}(f_{\bbi|_{n-1}}(x))\cdot
  \prod_{B\in\pi} f_{\bbi|_{n-1}}^{(|B|)}(x)\nonumber\\
  &=\sum_{\pi\in\Pi_{k+1}} \sum_{n=1}^\infty \phi_{i_n}^{(|\pi|)}(f_{\bbi|_{n-1}}(x))\cdot
  \prod_{B\in\pi}
  g_{|B|-1}(H_{\bi|_{n-1}}^{(|B|-2)}(x),\dots,H_{\bi|_{n-1}}(x))\cdot(f'_{\bbi|_{n-1}}(x))^{|\pi|}\nonumber
\end{align}
which concludes the proof.
\end{proof}
\begin{lemma}\label{thm:kbound}
  There exists $C_k$ such that for all $x\in I$ and all $\bi\in\Sigma$,
  \[
    |H_{\bi}^{(k)}(x)|\leq C_k.
  \]
\end{lemma}
\begin{proof}
  By induction, let 
  \[
    D_k:=\max_{k\in\Sigma_1}\max_{x\in[0,1)} |\phi_k^{(k)}(x)|
  \]
  and so
  $|H_{\bi}(x)| \leq D_1/(1-c_{\max})=:C_0$.
  Suppose that the statement is true for $k$, let 
  \[
    E_k:= \sup_{\substack{y_{j+1}\in[-C_j,C_j]\\0\,\leq\, j \,\leq\, k-1}}g_k(y_k,\dots,y_1).
  \]
  By \cref{thm:ugly},
  \[
    |H_{\bi}^{(k)}(x)| \leq \sum_{\pi\in\Pi_{k+1}}\sum_{n=1}^\infty D_{|\pi|} \cdot \prod_{B\in\pi}
    E_{|B|-1} \cdot c_{\max}^{(n-1)|\pi|}
    = \sum_{\pi\in\Pi_{k+1}} \frac{D_{|\pi|}\prod_{B\in\pi}E_{|B|-1}}{1-c_{\max}^{|\pi|}}
  \]
  and the statement follows.
\end{proof}
\begin{corollary}
  \label{thm:difcor}
  For all $k\geq 1$, for all $x,y\in I$ and for all $\bi\in\Sigma$,
  \[
    |H_{\bi}^{(k)}(x) - H_{\bi}^{(k)}(y)| \leq C_{k+1}\cdot |x-y|
  \]
\end{corollary}
We show the following useful H\"older type bound.
\begin{lemma}\label{thm:difbound}
  For all $k\geq 0$, $x\in I$ and $\bi,\bj\in\Sigma$ ,
  \[
    |H_{\bi}^{(k)}(x) - H_{\bj}^{(k)}(x)| \leq 2C_k c_{\max}^{|\bi\wedge\bj|},
  \]
  where $C_k>0$ are as defined in \cref{thm:kbound}.
\end{lemma}
\begin{proof}
  Let $m = |\bi\wedge\bj|$. Again, by \cref{thm:ugly,thm:kbound},
  \begin{align}
    &|H_{\bi}^{(k)}(x) - H_{\bj}^{(k)}(x)| 
    \nonumber\\
    =&
    \left|
    \sum_{\pi\in\Pi_{k+1}}\sum_{n=m+1}^\infty \phi_{i_n}^{(|\pi|)}(f_{\bbi|_{n-1}}(x))\cdot
    \prod_{B\in\pi} g_{|B|-1}(H_{\bi|_{n-1}}^{(|B|-2)}(x),\dots,H_{\bi|_{n-1}}(x)) \cdot
    f_{\bbi|_{n-1}}'(x)^{|\pi|}
  \right.
  \nonumber\\
     &
     \left. -
       \sum_{\pi\in\Pi_{k+1}}\sum_{n=m+1}^\infty \phi_{j_n}^{(|\pi|)}(f_{\bbj|_{n-1}}(x))\cdot
       \prod_{B\in\pi} g_{|B|-1}(H_{\bj|_{n-1}}^{(|B|-2)}(x),\dots,H_{\bj|_{n-1}}(x)) \cdot
       f_{\bbj|_{n-1}}'(x)^{|\pi|}
       \right|\nonumber\\
     &\leq
     \sum_{\pi\in\Pi_{k+1}}\sum_{n=m+1}^\infty 2D_{|\pi|}\cdot \prod_{B\in\pi} E_{|B|-1}
     \cdot c_{\max}^{n|\pi|} \leq 2 C_{k} c_{\max}^m.\nonumber
     \qedhere
  \end{align}
\end{proof}
We proceed with proving our main theorem, \cref{thm:main}.





\subsection{Proof of \texorpdfstring{\cref{thm:main}}{Main Theorem}}
We will prove our main theorem by contradiction and suppose that $\Phi$ is superexponentially
concentrated, that is there exists a sequence
$(\eta_n)_n$ such that $\log(\eta_n)/n\to-\infty$ and there exist a subsequence $n_{\ell}\in\bbN$ there exist distinct
words $\bi,\bj\in\Sigma_{n_\ell}$ such that
\[
  \sup_{x\in[0,1]}|f_{\bi}(x)-f_{\bj}(x)| \leq \eta_{n_\ell}.
\]
By \cref{thm:kbound},
\[
  |f_{\bi}''(x)| = |H_{\bi}(x)\cdot f_{\bi}'(x)|\leq C_0 \cdot c_{\max}^{n_\ell}
\]
and by \cref{thm:analyticity},
\[
  \sup_{x\in[0,1]}|f_{\bi}'(x) - f_{\bj}'(x)| \leq (2+C_0 c_{\max}^{n_\ell})\sqrt{\eta_{n_\ell}}.
\]
By the definition of $E_k$ and \cref{eq:fandg},
\[
  |f_{\bi}'''(x)| = |g_2(H_{\bi}'(x),H_{\bi}(x))\cdot f_{\bi}'(x)| \leq E_2\cdot c_{\max}^{n_\ell}.
\]
Combining them gives
\[
  \sup_{x\in[0,1]}|f_{\bi}''(x) - f_{\bj}''(x)| \leq (2+E_2c_{\max}^{n_\ell})\sqrt{2+C_0
  c_{\max}^{n_\ell}}\cdot
  \eta_{n_\ell}^{1/4}.
\]
We obtain
\begin{align*}
  \left|\frac{f_{\bi}''(x)}{f_{\bi}'(x)} - \frac{f_{\bj}''(x)}{f_{\bj}'(x)}\right|
    &\leq
    \frac{|f_{\bi}''(x)|}{|f_{\bi}(x)||f_{\bj}'(x)|}\cdot|f_{\bi}'(x) - f_{\bj}'(x)|
    +\frac{1}{|f'_{\bj}(x)|} \cdot |f_{\bi}''(x) - f_{\bj}''(x)|
    \\
    & 
    \leq \frac{C_0}{c_{\min}^{n_\ell}}(2+C_0c_{\max}^{n_\ell})\cdot \eta_{n_\ell}^{1/2}
    +\frac{1}{c_{\min}^{n_\ell}}(2+E_2
    c_{\max}^{n_\ell})\sqrt{2+C_0 c_{\max}^{n_\ell}} \cdot \eta_{n_\ell}^{1/4} \;=:\;\eta_{n_\ell}'.
\end{align*}
For such $\bi\neq\bj$, let $m = \max\{m\leq {n_\ell} : i_m \neq j_m\}$. Then $\bi = \bi'\bu$ and $\bj =
\bj'\bu$ with $|\bu| = m-1$ and $i_m\neq j_m$.

{\color{red}We stopped here}
By definition,
\[
  H_{\bbi}(x) - H_{\bbj}(x) = f_{\bu}'(x)\cdot (H_{\bbi'}(f_{\bu}(x)) - H_{\bbj'}(f_{\bu}(x)))
\]
and so
\[
  |H_{\bbi}(x) - H_{\bbj}(x)| \leq \frac{1}{|f'_{n_\ell}(x)|} \eta_{n_\ell}' \leq c_{\min}^{-{n_\ell}}
  \cdot \eta_{n_\ell}'
  =:\eta_{n_\ell}'' \leq \eta''_{|\bbi|}
\]
In particular, for every $n\geq 1$ there exists $1\leq m\leq n$, $\bi^{(n)},\bj^{(n)}\in\Sigma_m$,
and $\bu^{(n)}\in\Sigma_{n-m}$ such that $i_1 \neq j_1$ and
\[
  |H_{\bi^{(n)}}(f_{\bu^{(n)}}(x)) - H_{\bj^{(n)}}(f_{\bu^{(n)}}(x))| \leq \eta''_n
\]
for all $x\in I$.

We now claim that $|\bu^{(n)}| \to \infty$ and argue by contradiction, i.e.\ we assume there
exists a constant $C>0$ and infinitely many $n$ such that $\bu^{(n)} \leq C$. 
Since $|\bi^{(n)}|+ |\bu^{(n)}| = |\bj^{(n)}+|\bu^{(n)}| = n$ we conclude, by compactness, that
there exists a subsequence $n_l$ such that $\bi^{(n_l)} \to \bi^*\in\Sigma$,
$\bj^{(n_l)}\to\bj^*\in\Sigma$ and $\bu^{(n_l)}\to \bu^*\in\Sigma$ with $i_1^*\neq j_1^*$.
By \cref{thm:difbound} and \cref{thm:difcor} and that 
\[
  H_{\bi^*}(f_{\bu^*}(x))\equiv H_{\bj^*}(f_{\bu^*}(x)) 
\]
for all $x\in I$, using analyticity gives $H_{\bi}(x) \equiv H_{\bj}(x)$ for all $x\in I$. This
contradicts the main assumption.

Now combining \cref{thm:analyticity} and \cref{thm:kbound} we conclude that for all $k$ there
exists $\widetilde{C}_k>0$ such that for all $n\geq 1$ we have
\[
  |H_{\bi^{(n)}}^{(k)}(f_{\bu^{(n)}}(x)) - H_{\bj^{(n)}}^{(k)}(f_{\bu^{(n)}}(x))|
  \leq \widetilde{C}_k \cdot \frac{1}{(f'_{\bu^{(n)}}(x))^k}\cdot\eta_n^{1/2^k}
\]
for all $x\in[0,1]$.
By compactness, there exists $\bu^*\in\Sigma$ as well as $\bi^*,\bj^*\in\Sigma\cup\Sigma_*$ and a
subsequence $n_l$ such that $\bi^{(n_l)}\to \bi^*$ and $\bj^{(n_l)}\to \bj^*$ as well as
$f_{\bu^{(n_l)}}(x) \to \pi(\bu^*)$, where $i_1^*\neq j_1^*$ and
$H_{\bi^*}^{(k)}(\pi(\bu^*))=H_{\bj^*}^{(k)}(\pi(\bu^*))$ for all $k$.
Since $H_{\bi^*}$ and $H_{\bj^*}$ are analytic we get $H_{\bi^*}(x)\equiv H_{\bj^*}(x)$ for all
$x\in[0,1]$. However, this claim also contradicts our main assumption, proving the main theorem.
\qed

\section{More things}
We now continue with some more statements that follow from our main theorem, and how it relates to
other notions.
We define
\begin{definition}
  We say that the analytic IFS $\Phi = \{f_i\}_{i\in\Sigma_1}$ is \emph{conjugated} to a
  self-similar IFS if there exists an analytic, invertible $g:[0,1]\to\bbR$ and $\lambda_i
  \in(-1,1), t_i\in\bbR$ with $i\in\Sigma_1$ such that 
  \[
    f_{j}(x) = g^{-1}(\lambda_j g(x) + t_j)
  \]
  for all $j\in\Sigma_1$.
\end{definition}
Similarly,
\begin{definition}
  We say that the analytic IFS $\Phi = \{f_i\}_{i\in\Sigma_1}$ is \emph{sub-conjugated} to a
  self-similar IFS if there exist distinct words $\bi,\bj\in\Sigma_*$ such that 
  $\{f_{\bi},f_{\bj}\}$ is conjugated to a self-similar IFS.
\end{definition}
Our second main theorem is
\begin{theorem}
  Let $\Phi$ be sub-conjugated to a self-similar IFS. Then there exist $\bi,\bj\in\Sigma$ with
  $i_1\neq j_1$ such that, for all $x,y\in[0,1]$,
  \[
    H_{\bi}(x) \equiv H_{\bj}(x).
  \]
\end{theorem}
\begin{proof}
  By considering subsystems, we may assume without loss of generality that $\Phi =
  \{f_i\}_{i\in\Sigma_1}$ is conjugated to a self-similar IFS.
  We show that $H_{\bi}(x) \equiv H_{\bj}(x)$ for all $\bi,\bj\in\Sigma$.
  
  Let $p_{\bi}$ be the fixed point for some $\bi\in\Sigma_*$. Then,
  \[
    g(f_{\bi}(x)) = \lambda_{\bi} g(x) + t_{\bi}
    \quad\text{and}\quad
    g(p_{\bi}) = \frac{t_{\bi}}{1-\lambda_{\bi}}.
  \]
  We have
  \[
    g'(f_{\bi}(x))\cdot f_{\bi}'(x) = \lambda_{\bi} g'(x)
    \quad\text{and}\quad
    |g'(p_{\bi})||f'_{\bi}(p_{\bi})-\lambda_{\bi}| = 0.
  \]
  Since $|g'(p_{\bi})|>0$ we must have $f'_{\bi}(p_{\bi}) = \lambda_{\bi}$.
  Differentiating again we get
  \[
    g''(f_{\bi}(x))\cdot f'_{\bi}(x)^2 +g'(f_{\bi}(x))\cdot f''_{\bi}(x) = \lambda_{\bi} g''(x)
  \]
  and
  \[
    \frac{g''(p_{\bi})}{g'(p_{\bi})} =
    \frac{f_{\bi}''(p_{\bi})}{f_{\bi}'(p_{\bi})(1-f'_{\bi}(p_{\bi}))}
    =\frac{H_{\bbi}(p_{\bi})}{1-f'_{\bi}(p_{\bi})}.
  \]
  Now let $\bi,\bj\in\Sigma$ be arbitrary but fixed. Let $\bk_n = \bi|_n \bbj|_n$, where $\bbj|_n =
  j_nj_{n-1}\dots j_1$.
  Then $p_{\bk_n}\to \pi(\bi)$ as $n\to\infty$ and so
  \[
    \frac{g''(p_{\bk_n})}{g'(p_{\bk_n})} 
    =\frac{H_{\bj|_n \bbi|_n}(p_{\bk_n})}{1-f'_{\bk_n}(p_{\bk_n})}
    \to
    H_{\bj}(\pi(\bi))\quad \text{as}\quad n\to\infty.
  \]
  However, the left-hand side converges to $g''(\pi(\bi))/g'(\pi(\bi))$ and so we get $H_{\bj}(x) =
  H_{\bk}(x)$ for any choice of $\bj,\bk$. Since $\bi$ was also arbitrary, we get $H_{\bj}(x) =
  H_{\bk}(x)$ on an uncountable set and by analyticity, $H_{\bi}(x)\equiv H_{\bk}(x)$ for all $x\in I$
  and all $\bj,\bk\in \Sigma\cup \Sigma_*$.
\end{proof}

\section{Remarks}



\begin{thebibliography}{99}
  \bibitem{Rudin87}
  W. Rudin. 
  \textit{Real and complex analysis.}
  McGraw-Hill Book Co., New York, third edition, (1987).

  \bibitem{Hutchinson}
  Hutchinson??

  \bibitem{Falconer}
  Falconer??

\end{thebibliography}

\end{document}
